% Rerefence:
% - https://github.com/fnshr/latex-templates
% - https://okumuralab.org/tex/mod/forum/discuss.php?d=2661
% - https://ctan.math.illinois.edu/macros/latex/contrib/kanbun/kanbun-ja.pdf
\documentclass[ % ドキュメントクラス
lualatex, % LuaLaTeXを使う
tate, % 縦書き
% twocolumn, % 二段組
hanging_punctuation, % ぶら下げ組
paper=b6, % 用紙サイズ
reference_mark=interlinear, %注の合標を行間に
sidenote_length=10zw,       %sidenoteのために必要
book % 書籍のためのスタイル
]{jlreq}

%% フォント関連
\usepackage{luatexja-fontspec}
\usepackage[deluxe]{luatexja-preset}
\usepackage{luatexja-otf}

% Reference: https://qiita.com/isari/items/1d0b60b76c7ef168e376
\usepackage{newunicodechar}
\makeatletter
\chardef\my@J@horizbar="2015% Unicodeの2015
\newunicodechar{―}{\x@my@dash}
\def\x@my@dash{\@ifnextchar―{%
    \my@J@horizbar\kern-.5\zw\my@J@horizbar\kern-.5\zw}{%
    \my@J@horizbar}}
% 次が―なら2回目のkernまでを、そうでないなら普通の―を出力
\makeatother

%% 図表など
% 図の読みこみのために
\usepackage[hiresbb]{graphicx, xcolor}
\usepackage{booktabs} % 表の横罫線

%% 囲み枠
\usepackage{tcolorbox}
\tcbuselibrary{breakable} % ページをまたいで分割できるように

%% misc
% \usepackage{okumacro} % Lualatexではだめ
% \usepackage{plext}    % Lualatexではだめ(傍点 美文書7 p305 傍注参照)
\usepackage{lltjext}    % for LuaLaTeX
\usepackage{pxrubrica}  % ルビをつける(okumacroのrubyは使わない)
% \usepackage[prefix=]{gckanbun}   % https://konoyonohana.blog.fc2.com/blog-entry-610.html
\usepackage[kumi=aki,tateaki=1,yokoaki=4]{kanbun} % https://ctan.math.illinois.edu/macros/latex/contrib/kanbun/kanbun-ja.pdf
% \usepackage{sfkanbun} % 漢文
% sfkanbun パッケージは、
% http://xymtex.my.coocan.jp/fujitas2/texlatex/index.html
% から入手可能だが、lualatexでは使えない。
% https://okumuralab.org/tex/mod/forum/discuss.php?d=2655
% に使えるようにしたものが配布されている。→ まともに動かない

%% 見出しのスタイルの設定
% chapterの定義
\DeclareTobiraHeading{chapter}{1}{% chapter を扉見出しに
  format={\null\vfil {\huge\gtfamily\bfseries {\LARGE #1}#2}}, % 見出しのフォント
  label_format={第\thechapter 章\hspace{2\zw}} % ラベルのフォーマット
}
\DeclareBlockHeading{chapter}{1}{ % chapter を別行見出しに
  pagebreak=cleardoublepage, % 章を始める前に改丁
  label_format={第\thechapter 章}, % ラベルのフォーマット
  font={\gtfamily\LARGE}, % 見出しのフォント
  lines=3,after_lines=2, % 見出しのために5行取り、後ろの方が2行分広い
  indent=2\zw % インデント
}

% section の定義
\renewcommand{\thesection}{} % 節の番号はなしが基本
\DeclareBlockHeading{section}{2}{ % section を別行見出しに
  font={\gtfamily}, % 見出しのフォント
  lines=1, before_lines=1% 見出しの前に1行取る
}

%% 目次の設定
\setcounter{tocdepth}{1} % sectionまでを目次に

%% hyperrefの設定
\usepackage[%
unicode=true,%
pdftitle=タイトル, % PDFのタイトル
pdfauthor=作成者, % PDFの作成者
bookmarks=true, % PDFにしおりをつける
bookmarksnumbered=true, % しおりに節番号などをつける
colorlinks=false, % リンクには色をつけない
hyperfootnotes=false, % 脚注からのリンクを作らない
pdfborder={0 0 0}, % 枠なし
pdfdirection=R2L, % 開く方向
pdfpagelayout=TwoPageRight, % 奇数頁が右側になるような見開きモードで開く
pdfpagemode=UseNone
]{hyperref}

% PDFにしたときのしおりの文字化けを防ぐ
% \usepackage{pxjahyper}
% Lualatexではだめ

% hyperref を使っているときに
% 目次でのページ番号の向きを適切にする
\makeatletter
\def\contentsline#1#2#3#4{%
  \csname l@#1\endcsname{%
    \hyper@linkstart{link}{#4}{#2}
    \hyper@linkend}{\tatechuyoko{#3}}}
\makeatother
% lltjext package \rensuji{}は広すぎるので \tatechuyoko に変更.

\begin{document}
% maketitle を使わずに独自のタイトルページを作る
\begin{titlepage}
  \vspace*{10mm}
  \noindent{\fontsize{30pt}{48pt}\gtfamily\bfseries タイトル}
  \vfill

  \begin{flushright}
    {\gtfamily\bfseries\huge 著者 名}
  \end{flushright}
\end{titlepage}

\phantomsection
\addcontentsline{toc}{chapter}{序}
\chapter*{序}

ここには序の内容が入る。

\tableofcontents % 目次


\chapter{最初の章}

ここは最初の章の冒頭の文章が入る。
ここは最初の章の冒頭の文章が入る。


\section{最初の節の見出し}

ここは最初の節の文章が入る。
ここは最初の節の文章が入る。
ここは最初の節の文章が入る。
ここは最初の節の文章が入る。
ここは最初の節の文章が入る。


\section{第二の節の見出し}

ここは第二の節の文章が入る。
ここは第二の節の文章が入る。
ここは第二の節の文章が入る。
ここは第二の節の文章が入る。
ここは第二の節の文章が入る。


\chapter{便利な命令}

\section{文字装飾}

  \bou{傍点}・\kenten{圏点}・\kasen{傍線}

\section{特殊文字など}

  \ajMaru{0} \ajMaru{5} \ajMaru{42} \ajMaru{100}
  \ajMaru*{0} \ajMaru*{5} \ajMaru*{42} \ajMaru*{100}
  \ajKuroMaru{0} \ajKuroMaru{5} \ajKuroMaru{42} \ajKuroMaru{100}
  \ajKuroMaru*{0} \ajKuroMaru*{5} \ajKuroMaru*{42} \ajKuroMaru*{100}

  \ajKaku{0} \ajKaku{5} \ajKaku{42} \ajKaku{100}
  \ajKaku*{0} \ajKaku*{5} \ajKaku*{42} \ajKaku*{100}
  \ajKuroKaku{0} \ajKuroKaku{5} \ajKuroKaku{42} \ajKuroKaku{100}
  \ajKuroKaku*{0} \ajKuroKaku*{5} \ajKuroKaku*{42} \ajKuroKaku*{100}

  \ajMaruKaku{0} \ajMaruKaku{5} \ajMaruKaku{42} \ajMaruKaku{100}
  \ajMaruKaku*{0} \ajMaruKaku*{5} \ajMaruKaku*{42} \ajMaruKaku*{100}
  \ajKuroMaruKaku{0} \ajKuroMaruKaku{5} \ajKuroMaruKaku{42} \ajKuroMaruKaku{100}
  \ajKuroMaruKaku*{0} \ajKuroMaruKaku*{5} \ajKuroMaruKaku*{42} \ajKuroMaruKaku*{100}


「こら〳〵」「どれ〴〵」

参加者は\,\tatechuyoko{12}\,人だった。


\section{注釈}

脚注\sidenote{脚注。}を表示する。

後注\endnote{これが後注の文章である。}を表示する。

割注\warichu{これが割注の文章である。}を表示する。


\section{漢文}

\Kanbun
未(いま){ダ}‹ざ›«ル»[レ]知(し){ラ}[二]仁(じん)義(ぎ){ヲ}[一]也(なり)
\EndKanbun

\printkanbun

\Kanbun
此レ乃チ信(しん)之‘所―[三]以’(ゆゑん)為ル[二]陛下ノ禽(とりこ)ト[一]也。
\EndKanbun
\let\信\printkanbun
\Kanbun
孤之有ルハ[二]孔明[一],猶ホ‹ごと›«キ»[二]魚之有ルガ[一レ]水也。
\EndKanbun
\let\孔明\printkanbun

\孔明\par\bfseries\信

\end{document}
%%% Local Variables:
%%% mode: japanese-latex
%%% TeX-master: t
%%% End:
